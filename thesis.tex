\documentclass[a4paper,12pt,bibtotoc,titlepage, liststotoc,BCOR7mm,headsepline,pointlessnumbers]{scrbook}
\usepackage[numbers]{natbib} 
\usepackage[english]{babel}
%SHOW COMMENTS
\newcommand{\mycomment}[1]{\color{red}[COMMENT/TODO: \textbf{#1}]\color{black}}
%DONT SHOW COMMENTS
\renewcommand*{\chapterpagestyle}{useheadings}
%% Anpassen: %%%%%%%%%%%%%%%%%%%%%%%%%%%%%%
\newcommand{\upbautor}{Your Name} % Name des Autors
\newcommand{\upbtitel}{Title of Your Thesis} % Titel der Arbeit
\newcommand{\upbtyp}{Bachelor Thesis} % Bachelor oder Master 
\newcommand{\upbmatrikelnr}{1232808} % Matrikelnummer
\newcommand{\upbemail}{youremail@mail.com} % E-Mail des Autors
\newcommand{\upbzweitgutachter}{TODO} % Namen der Betreuer
\newcommand{\upbdate}{\today} % Datum auf der Arbeit (\today f�gt das aktuelle Datum ein)
\newcommand{\E}{\mbox{I\negthinspace E}}
\usepackage{times}

\usepackage[ansinew]{inputenc} % Zeichenkodierung
\usepackage[T1]{fontenc} % Trennung von W�rtern mit Umlauten
\usepackage{natbib} % BibTeX
\usepackage[algo2e, german, ruled, vlined]{algorithm2e}
\usepackage{algorithm}

\usepackage{array}
\usepackage{amsfonts} % Mathematische Zeichen
\usepackage{amssymb}  % Mathematische Zeichen
\usepackage{scrlayer-scrpage} % Komma für Kopfzeilen
\usepackage{color}    % Farbiger Text
\usepackage{graphicx} % Funktionen zum Einf�gen von Bildern
\usepackage{graphics}
\usepackage[plainpages=false,pdfpagelabels=false,citecolor=Black, linkcolor=Black]{hyperref} %Verweise werden Links im PDF
\usepackage{url}
\usepackage{mathtools}
\usepackage{sectsty}
\usepackage{listings}
\usepackage{verbatim}

\usepackage{theorem} % Theorem-Optionen
\theoremstyle{break} % Zeilenumbruch nach Theorem
\newtheorem{definition}{Definition}
\newtheorem{theorem}{Theorem}


\lstset { columns=fixed, basicstyle=\footnotesize \ttfamily, tabsize=2,
keywordstyle=\color[rgb]{0.00,0.00,0.50}{}\bfseries, commentstyle=\color[rgb]{0.00,0.50,0.25}{},
showstringspaces=false, numbers=left, numberstyle=\tiny, numbersep=10pt, xleftmargin=20pt,
language=JAVA }

\hypersetup{colorlinks=false,linkcolor=Black, pdfborder={1 1 0.1}}
\definecolor{Black}{rgb}{0.0,0.0,0.5}
\newenvironment{code}{\begingroup\footnotesize \verbatim}{\endverbatim\endgroup}

%========================= Eigene Konfigurationen =====================================
\def\MdR{\ensuremath{\mathbb{R}}}
\def\MdN{\ensuremath{\mathbb{N}}}
%======================================================================================
\parindent 0 cm
\sloppy

\typearea[current]{current}
\pagestyle{scrheadings} 

\addtokomafont{caption}{\small} % Aktive Auszeichnung von Legenden
\setkomafont{captionlabel}{\sffamily\bfseries}
\bibliographystyle{plain}


\title{\upbtitel}
\author{
\upbautor
}
\publishers{{\normalsize Supervisors:}
\upbbetreuer
}
\date{\upbdate}

\renewcommand{\thepage}{\roman{page}}
\usepackage[ ]{titlesec} 
\titleformat{\chapter}[display]
  { \normalsize \huge  \color{black}}
  {\flushright \normalsize \MakeUppercase { \chaptertitlename \hspace{1 ex} }  { \fontsize{60}{60}\selectfont \sffamily  \thechapter }} {10 pt}{\bfseries\huge}
\begin{document}

\begin{titlepage}
\begin{center}
\vspace*{-0.75cm}
{\bf{ \huge{\upbtitel} }} \\

\vspace*{1.0cm}


\large{\upbautor}  \\
\vspace*{0.5cm}
\Large{\today}\\
\vspace*{0.5cm}
\upbmatrikelnr \\
\vspace*{1.0cm}
\upbtyp\\
\vspace*{0.25cm}
\normalsize
at \\
\vspace*{0.25cm}
Algorithm Engineering Group Heidelberg \\
Heidelberg University \\
\vspace*{0.25cm}

\vspace*{0.75cm}
\begin{center}
You can insert a fancy image from your thesis here or remove this block.
\end{center}
\vspace*{0.5cm}

Supervisor:\\
Univ.-Prof. PD. Dr. rer. nat. Christian Schulz \\
 
\vspace*{0.50cm}
Co-Referee: \\
 \upbzweitgutachter               
%\end{tabular}

\end{center}
\end{titlepage}

\chapter*{Acknowledgments}
TODO

\vspace*{\fill}


Ich erkl\"are hiermit, dass ich die vorliegende Arbeit selbst\"andig verfasst und keine anderen als
die angegebenen Quellen und Hilfsmittel verwendet habe. \\ \\
Heidelberg, \today \\ \\  \\



\upbauthor

\pagebreak
	
\chapter*{Abstract}
\addcontentsline{toc}{chapter}{Abstract}
	
	\tableofcontents
	
	\mainmatter
	\newpage
\chapter{Introduction}
\section{Motivation}
\section{Our Contribution}
\section{Structure}
The remainder of this thesis is organized as follows.
	\newpage
\chapter{Fundamentals}
\section{General Definitions}
\chapter{Related Work}
\chapter{Name The Chapter, This is Your Core Chapter}
\chapter{Experimental Evaluation}
\chapter{Discussion}
\section{Conclusion}
\section{Future Work}
\appendix
\chapter*{Implementation Details}
\chapter*{Command-Line Arguments}
\chapter*{Further Results}
\chapter*{Summary}
\addcontentsline{toc}{chapter}{Abstract (German)}
\newpage
\backmatter
\bibliographystyle{natbib}
\bibliography{quellen}

\end{document}
